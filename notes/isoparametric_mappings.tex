\documentclass[10pt]{article}
\usepackage[margin=1in]{geometry}
\usepackage{amsmath, amssymb, bm, mathtools}
\usepackage{graphicx, subfig, wrapfig, mwe, epstopdf, pdfpages}
%\usepackage[section]{placeins}
\usepackage{enumerate}
\usepackage{stmaryrd}
\usepackage{algorithm, algpseudocode}
\usepackage{float}
\usepackage[hidelinks]{hyperref}
\usepackage[toc, nonumberlist]{glossaries}
\newglossary{symbol}{sym}{sbl}{List of Discretization Symbols}
\include{glossary}

% set paragraph indentation to be 0 by default
\setlength\parindent{0pt}

% macros
\newcommand{\bigO}[1]{\mathcal{O}\left(#1\right)}
\newcommand{\LL}[1]{L^{2}\left(#1\right)}
\newcommand{\LLW}{\LL{\Omega}}
\newcommand{\Hdiv}[1]{H^{\text{div}}\left(#1\right)}
\newcommand{\HdivW}{\Hdiv{\Omega}}
\newcommand{\Th}{\mathcal{T}_h}
\newcommand{\dTh}{\partial\mathcal{T}_h}
\newcommand{\uhath}{\widehat{u}_h}
\newcommand{\qhath}{\widehat{\bm{q}}_h}
\newcommand{\lims}[2]{\bigg\rvert^{#1}_{#2}}
\newcommand{\at}[1]{\lims{}{#1}}
\def\rddots#1{\cdot^{\cdot^{\cdot{#1}}}}

\newcommand{\LLN}[1]{\left\Vert #1 \right\Vert_{\LLW}}
\newcommand{\HN}[1]{\left\Vert #1 \right\Vert_{H^1(\Omega)}}
\newcommand{\HSN}[1]{\left| #1 \right|_{H^1(\Omega)}}
\newcommand{\Hb}{H^{1,\text{broken}}}
\newcommand{\HbW}{H^{1,\text{broken}}(\Omega)}

% DG integrals, jumps 
\newcommand{\Iint}[3]{\left(#1 ,\, #2 \right)_{#3}}
\newcommand{\IintK}[2]{\Iint{#1}{#2}{K}}
\newcommand{\Eint}[3]{\left\langle #1 ,\, #2 \right\rangle_{#3}}
\newcommand{\EintK}[2]{\Eint{#1}{#2}{\partial K}}
\newcommand{\jmp}[1]{\llbracket #1 \rrbracket}

% physics
\newcommand{\ReL}{\text{Re}_L}
\newcommand{\Pra}{\text{Pr}}
\newcommand{\Pe}{\text{Pe}}

% restriction operator
\newcommand\restr[2]{{ % we make the whole thing an ordinary symbol
  \left.\kern-\nulldelimiterspace % automatically resize the bar with \right
  #1 % the function
  \vphantom{\big|} % pretend it's a little taller at normal size
  \right|_{#2} % this is the delimiter
  }}

\title{Isoparametric Finite Element Mappings}
\author{Corbin Foucart}
\date{}
\begin{document}
\maketitle

\section{Isoparametric Mappings}

\subsection{mapping from reference element to physical space}
We express the transformation from reference space to physical space as a linear combination of the
basis functions of the chosen finite element space. The concise statement can be written:
\begin{align}
  \bm{x}(\bm{\xi}) = \sum_{j=1}^{n_b} \bm{x}_j^K \psi_j(\bm{\xi})
  \label{eq:isotrans}
\end{align}

Where $n_b$ is the number of nodal shape functions defined on the master element, and $\bm{x}_j^K$
are the interpolation points on the physical space triangle. Note that the number of interpolation
points in physical space must be the same as the number of basis functions. Therefore the
`interpolation points' are the nodal points in physical space. We can clarify the above expression
by writing the explicit statement in two dimensions:

\begin{align}
  x(\xi, \eta) = \sum_{j=1}^{n_b} x_j^K \psi_j(\xi, \eta) \\
  y(\xi, \eta) = \sum_{j=1}^{n_b} y_j^K \psi_j(\xi, \eta)
\end{align}

So if we would like to map any set of points (here we choose quadrature points $\bm{\xi}_q$, but the
approach is general and can be used for any set of points $\bm{\xi}_\alpha$ defined on the master
element) defined on the master element to their corresponding points in physical space, we can use
the mapping directly for each quadrature point $\bm{\xi}_q$:

\begin{align}
  \bm{x}(\bm{\xi}_q) = \sum_{j=1}^{n_b} \bm{x}_j^K \psi_j(\bm{\xi}_q) 
  \qquad\rightarrow\qquad
  \underbrace{
  \begin{bmatrix}
    \psi_1(\bm{\xi}_{1}), & \cdots, & \psi_{n_b}(\bm{\xi}_1) \\
    \vdots & & \vdots  \\
    \psi_1(\bm{\xi}_{q}), & \cdots, & \psi_{n_b}(\bm{\xi}_q)
  \end{bmatrix}}_{G}
  \begin{bmatrix}
     x_1^K, & y_1^K\\
     & \\
     \vdots & \vdots  \\
     & \\
     x_{n_b}^K, & y_{n_b}^K\\
   \end{bmatrix}
\end{align}

In order to map any set of points on the master element $\bm{\xi}_\alpha$ (indexed here by
$\alpha$), we need the values of the nodal shape functions evaluated at those points
$\bm{\xi}_\alpha$ and the physical space nodal points $\bm{x}_j^K$. Note that in the case of mapping
the nodal points on the master element to the physical space nodes, note that the transformation
matrix $G$ is simply the identity matrix $I$ (nodal basis), and the physical space nodal points are
recovered.

\subsection{computation of mapping derivatives, jacobian $J$}

In this subsection, we use the explicit example of the derivatives of the isoparametric
transformation in 2D, for clarity. The key idea is that since we have an explicit representation of
our transformation in (\ref{eq:isotrans}), we can take the derivative of the transformation with
respect to master element coordinates $\xi, \eta$

\begin{align}
  \frac{\partial x}{\partial \xi} =
  \frac{\partial x(\xi,\eta)}{\partial \xi}
  = \frac{\partial }{\partial \xi}\left[\sum_{j=1}^{n_b} x_j^K \psi_j(\xi, \eta)\right]
  =\sum_{j=1}^{n_b} x_j^K \frac{\partial \psi_j}{\partial \xi}
  \label{eq:dxdxi}
\end{align}

The last expression of (\ref{eq:dxdxi}) is easily computable since we know both the derivates of the
shape functions $\psi_j$ on the master element with respect to the master coordinates $\xi, \eta$.
The Jacobian matrix of the transformation from the reference element $\widehat{K}$ to the physical
space element $K$ at a specific point $\bm{x}_\alpha$ with corresponding point $\bm{\xi}_\alpha$ on
the master element can be written

\begin{align}
  J\big|_{\bm{x}_\alpha}
  =
  \begin{bmatrix}
    \frac{\partial x}{\partial \xi} & \frac{\partial y}{\partial \xi} \\
    \frac{\partial x}{\partial \eta} & \frac{\partial y}{\partial \eta} \\
  \end{bmatrix}_{\bm{x}_\alpha}
  =
  \begin{bmatrix}
    \sum\limits_{j=1}^{n_b} x_j^K \frac{\partial \psi_j(\bm{\xi}_\alpha)}{\partial \xi} &
    \sum\limits_{j=1}^{n_b} y_j^K \frac{\partial \psi_j(\bm{\xi}_\alpha)}{\partial \xi} \\
    \sum\limits_{j=1}^{n_b} x_j^K \frac{\partial \psi_j(\bm{\xi}_\alpha)}{\partial \eta} &
    \sum\limits_{j=1}^{n_b} y_j^K \frac{\partial \psi_j(\bm{\xi}_\alpha)}{\partial \eta} \\
  \end{bmatrix}
  %\qquad\rightarrow\qquad
  \;\rightarrow\;
  \begin{bmatrix}
    \frac{\partial \psi_1(\bm{\xi}_\alpha)}{\partial \xi}
    & \cdots &
    \frac{\partial \psi_{n_b}(\bm{\xi}_\alpha)}{\partial \xi}\\
    \frac{\partial \psi_1(\bm{\xi}_\alpha)}{\partial \eta}
    & \cdots &
    \frac{\partial \psi_{n_b}(\bm{\xi}_\alpha)}{\partial \eta}\\
  \end{bmatrix}
  \begin{bmatrix}
     x_1^K, & y_1^K\\
     & \\
     \vdots & \vdots  \\
     & \\
     x_{n_b}^K, & y_{n_b}^K\\
   \end{bmatrix}
   \label{eq:Jcomp}
\end{align}

Note that (\ref{eq:Jcomp}) is directly computable from the derivatives of the shape functions,
evaluated at the point of interest $\bm{\xi}_\alpha$ on the master element, as well as the physical
nodal points. In practice, we evaluate the derivatives of the nodal shape functions at the
quadrature points and nodal points (for quadrature-based and quadrature-free integrals,
respectively).  We will denote $\det(J)$ as $|J|$. For every point pairing $\bm{\xi}_\alpha,\,
\bm{x}_\alpha$ on the reference element and physical space (respectively), once we have computed
$J(\bm{x}_\alpha)$, computing $|J|$ at $\bm{x}_\alpha$ is simply a matter of computing the
determinant. \\


\subsection{computation of inverse mapping derivatives, $J^{-1}$}

We will explicitly need to compute the derivates of the inverse mapping from physical space to
reference space (discussed in \ref{ssec:integral_operators}) which arises from the need to compute
terms of the form $\frac{\partial \phi_i}{\partial x}$, i.e., the derivatives of the global nodal
basis functions with respect to physical space coordinates. The difficulty is that computing and
storing the derivatives of all global basis functions at every point of interest in the
computational domain would be enormously expensive. Instead, we wish to map the derivatives of the
global basis functions to the derivatives of the shape functions on the master element, e.g.,
$\frac{\partial \psi_i(\bm{\xi})}{\partial \xi}$, which are known and stored. We can link the two
via the chain rule. In one spatial dimension, (\ref{eq:isotrans}) can be applied with the chain rule
to conclude

\begin{align}
  \frac{\partial \phi_i}{\partial \xi} 
  = \frac{\partial \phi_i}{\partial x} \frac{\partial x}{\partial \xi}
\end{align}

Since we have the functional dependence $\phi_i(x(\xi))$. In two dimensions, we have the functional
dependence 

\begin{figure}[thp!]
  \centering \includegraphics[width=0.3\textwidth]{img/func_dependence.pdf}
\end{figure}

Hence we can write the chain rule statement applied over the isoparametric mapping as 
\begin{align}
  \frac{\partial \phi_i}{\partial \xi}
  = \frac{\partial \phi_i}{\partial x} \frac{\partial x}{\partial \xi}
  + \frac{\partial \phi_i}{\partial y} \frac{\partial y}{\partial \xi}, \qquad
  \frac{\partial \phi_i}{\partial \eta}
  = \frac{\partial \phi_i}{\partial x} \frac{\partial x}{\partial \eta}
  + \frac{\partial \phi_i}{\partial y} \frac{\partial y}{\partial \eta}
\end{align}

\begingroup
\renewcommand*{\arraystretch}{1.5}
\begin{align}
  \Rightarrow
  \begin{bmatrix}
    \frac{\partial \phi_i}{\partial \xi} \\
    \frac{\partial \phi_i}{\partial \eta}
  \end{bmatrix}
  =
  \begin{bmatrix}
    \frac{\partial x}{\partial \xi} & \frac{\partial y}{\partial \xi} \\
    \frac{\partial x}{\partial \eta} & \frac{\partial y}{\partial \eta} \\
  \end{bmatrix}
  \begin{bmatrix}
    \frac{\partial \phi_i}{\partial x} \\
    \frac{\partial \phi_i}{\partial y} \\
  \end{bmatrix}
  =J
  \begin{bmatrix}
    \frac{\partial \phi_i}{\partial x} \\
    \frac{\partial \phi_i}{\partial y} \\
  \end{bmatrix}
\end{align}
\endgroup

Where $J$ is the jacobian matrix computed at a physical space point $\bm{x}_\alpha$ in the previous
section. We can directly invert this expression to find the inverse transform $J^{-1}$, which will
allow us to compute the derivatives of the global basis functions in terms of the derivatives of the
master element shape functions.

\begingroup
\renewcommand*{\arraystretch}{1.5}
\begin{align}
  \begin{bmatrix}
    \frac{\partial \phi_i}{\partial x} \\
    \frac{\partial \phi_i}{\partial y} \\
  \end{bmatrix}
  =
  \begin{bmatrix}
    \frac{\partial \xi}{\partial x} & \frac{\partial \xi}{\partial y} \\
    \frac{\partial \eta}{\partial x} & \frac{\partial \eta}{\partial y} \\
  \end{bmatrix}
  \begin{bmatrix}
    \frac{\partial \phi_i}{\partial \xi} \\
    \frac{\partial \phi_i}{\partial \eta}
  \end{bmatrix}
  =J^{-1}
  \begin{bmatrix}
    \frac{\partial \phi_i}{\partial \xi} \\
    \frac{\partial \phi_i}{\partial \eta}
  \end{bmatrix}
\end{align}
\endgroup

Which can be explicitly written as 
\begin{align}
    \begin{split}
    \frac{\partial \phi_i}{\partial x}
    &= \frac{\partial \phi_i}{\partial \xi} J^{-1}_{11}
    + \frac{\partial \phi_i}{\partial \eta} J^{-1}_{12} \\
    \frac{\partial \phi_i}{\partial y}
    &= \frac{\partial \phi_i}{\partial \xi} J^{-1}_{21}
    + \frac{\partial \phi_i}{\partial \eta} J^{-1}_{22}
    \end{split}
    \label{eq:def_inv_der}
\end{align}

Extension to the general case should be apparent from the specific two-dimensional examples.
Discrete computation of these $\frac{\partial \bm{\xi}}{\partial \bm{x}}$ operators will allow us to
perform differentiation and integration on the master element rather than in physical space, which
would be prohibitively expensive.

\subsection{continuous integral operators} \label{ssec:integral_operators}

We can write the forms of the continuous integral operators by directly substituting the inverted
chain rule relations (\ref{eq:def_inv_der}) into the physical space integral expressions.

  \subsubsection{interior operators}

  Weak laplacian:
  \begin{align}
    \begin{split}
    \int_{K}^{} \nabla\phi_j\cdot\nabla\phi_i \,d\bm{x}
    &= \int_{K}^{} \left( \frac{\partial \phi_i}{\partial x} \frac{\partial \phi_j}{\partial x}
    + \frac{\partial \phi_i}{\partial y} \frac{\partial \phi_j}{\partial y}\right)\,d\bm{x} \\
    &=
    \int_{\widehat{K}} 
    \left( \frac{\partial \widehat{\phi_i}}{\partial \xi} J^{-1}_{11} 
    + \frac{\partial \widehat{\phi_i}}{\partial \eta} J^{-1}_{12} \right) 
    \left( \frac{\partial \widehat{\phi_j}}{\partial \xi} J^{-1}_{11} 
    + \frac{\partial \widehat{\phi_j}}{\partial \eta} J^{-1}_{12} \right) 
    |J|\,d\bm{\widehat{x}} \\
    &\qquad + 
    \int_{\widehat{K}} 
    \left( \frac{\partial \widehat{\phi_i}}{\partial \xi} J^{-1}_{21} 
    + \frac{\partial \widehat{\phi_i}}{\partial \eta} J^{-1}_{22} \right) 
    \left( \frac{\partial \widehat{\phi_j}}{\partial \xi} J^{-1}_{21} 
    + \frac{\partial \widehat{\phi_j}}{\partial \eta} J^{-1}_{22} \right) 
    |J|\,d\bm{\widehat{x}} \\
    \end{split}
    \label{eq:cts_weak_laplacian}
  \end{align}

  Convection-like derivatives:
  \begin{align}
    \begin{split}
      \int_{K}^{} \nabla\phi_i\cdot \bm{\phi}_j \,d\bm{x}
    &= \int_{K}^{} \left( \frac{\partial \phi_i}{\partial x} \phi_j
    + \frac{\partial \phi_i}{\partial y} \phi_j \right)\,d\bm{x} \\
    &= \int_{\widehat{K}} 
    \left( \frac{\partial \widehat{\phi_i}}{\partial \xi} J^{-1}_{11} 
    + \frac{\partial \widehat{\phi_i}}{\partial \eta} J^{-1}_{12} \right) \phi_j
    |J|\,d\bm{\widehat{x}} 
    + \int_{\widehat{K}} 
    \left( \frac{\partial \widehat{\phi_i}}{\partial \xi} J^{-1}_{21} 
    + \frac{\partial \widehat{\phi_i}}{\partial \eta} J^{-1}_{22} \right) \phi_j
    |J|\,d\bm{\widehat{x}} \\
    \end{split}
    \label{eq:cts_weak_convection}
  \end{align}

  Source term integration:

  \begin{align}
    \int_{K}^{} f_\Omega \phi_i\, d\bm{x} 
    = \int_{\widehat{K}}^{} \widehat{f}_\Omega\, \widehat{\phi_i} |J| \, d\bm{\widehat{x}}
    = \int_{\widehat{K}}^{} f_\Omega(\bm{\xi}(\bm{x})) \widehat{\phi_i} |J| \, d\bm{\widehat{x}}
  \end{align}

  Where there is some subtlety in $\widehat{f}_\Omega$, since we are integrating the
  $f_\Omega(\bm{x})$ over some $K$ mapped to the reference element $\widehat{K}$. 
  This inverse mapping $\bm{\xi}(\bm{x})$ maps physical space locations to their corresponding
  locations on the master element. While in general, this inverse mapping is equivalent to solving a
  non-linear set of equations, in practice, we don't need to compute the inverse map, because the
  points at which we evaluate the function in physical space (either nodal points or quadrature
  points) are mapped from the master element nodal or quadrature points -- i.e., the points on the
  master element which correspond to the physical space nodal/quadrature points are simply the
  nodal/quadrature points on the master element, by construction!

\subsection{discrete integral operators} \label{ssec:disc_integral_operators}
  \subsubsection{quadrature-based operators}

  We can choose a classical quadrature scheme (points and weights defined on the master element) and
  approximate the continuous form of the weak laplacian operator (\ref{eq:cts_weak_laplacian})
  accordingly. We done the master element quadrature points by $\bm{\xi}_q$ and corresponding mapped
  physical space quadrature points $\bm{x}_q$.\\

  Weak laplacian:
  \begin{align}
    \begin{split}
    &\int_{K}^{} \nabla\phi_j\cdot\nabla\phi_i \,d\bm{x} 
    \approx
    \sum_{q=1}^{n_q}
    \left( \frac{\partial \widehat{\phi_i}}{\partial \xi}(\bm{\xi}_q) J^{-1}_{11}(\bm{x}_q)
    + \frac{\partial \widehat{\phi_i}}{\partial \eta}(\bm{\xi}_q) J^{-1}_{12}(\bm{x}_q) \right) 
    \left( \frac{\partial \widehat{\phi_j}}{\partial \xi}(\bm{\xi}_q) J^{-1}_{11}(\bm{x}_q)
    + \frac{\partial \widehat{\phi_j}}{\partial \eta}(\bm{\xi}_q) J^{-1}_{12}(\bm{x}_q) \right) 
    w_q |J|\bigg|_{\bm{x}_q}\\
    &\qquad+ 
    \sum_{q=1}^{n_q}
    \left( \frac{\partial \widehat{\phi_i}}{\partial \xi}(\bm{\xi}_q) J^{-1}_{21}(\bm{x}_q)
    + \frac{\partial \widehat{\phi_i}}{\partial \eta}(\bm{\xi}_q) J^{-1}_{22}(\bm{x}_q) \right) 
    \left( \frac{\partial \widehat{\phi_j}}{\partial \xi}(\bm{\xi}_q) J^{-1}_{21}(\bm{x}_q)
    + \frac{\partial \widehat{\phi_j}}{\partial \eta}(\bm{\xi}_q) J^{-1}_{22}(\bm{x}_q) \right) 
    w_q|J|\bigg|_{\bm{x}_q}
    \end{split}
    \label{eq:disc_weak_laplacian_quad}
  \end{align}

  Convection-like:
  \begin{align}
    \begin{split}
      \int_{K}^{} \nabla\phi_i\cdot \bm{\phi}_j \,d\bm{x}
    &\approx \sum_{q=1}^{n_q}
    \left( \frac{\partial \widehat{\phi_i}}{\partial \xi}(\bm{\xi}_q) J^{-1}_{11}(\bm{x}_q)
    + \frac{\partial \widehat{\phi_i}}{\partial \eta}(\bm{\xi}_q) J^{-1}_{12}(\bm{x}_q) \right)
    \phi_j(\bm{\xi}_q) w_q|J|\bigg|_{\bm{x}_q} \\
    & \qquad + \sum_{q=1}^{n_q}
    \left( \frac{\partial \widehat{\phi_i}}{\partial \xi}(\bm{\xi}_q) J^{-1}_{21}(\bm{x}_q)
    + \frac{\partial \widehat{\phi_i}}{\partial \eta}(\bm{\xi}_q) J^{-1}_{22}(\bm{x}_q) \right)
    \phi_j(\bm{\xi}_q) w_q |J|\bigg|_{\bm{x}_q}
    \end{split}
    \label{eq:disc_weak_convection_quad}
  \end{align}

  \subsubsection{quadrature-free operators}

  It is possible to form integral operators without using a ``traditional'' quadrature scheme (points
  and weights chosen on the master element distinct from the nodal points), such as described in
  \cite{ueckermann_lermusiaux_JCP2016}. In the discussion above, we were careful to describe the
  isoparametric mapping at arbitrary spatial locations.  Instead of computing the $\det J$ and
  $J^{-1}$ quantities at the quadrature points in physical space, we compute the data at the nodal
  points only. \\

  Because the scheme in \cite{ueckermann_lermusiaux_JCP2016} is used in a mixed-method finite
  element formulation (HDG) using the so-called ``strong form'' DG-FEM, the weak laplacian as
  written above does not appear. Instead, the integral derivative operators are similar to the
  convection-like operators in (\ref{eq:cts_weak_convection}). \\

  Convection-like:
  \begin{align}
    \begin{split}
      \int_{K}^{} \nabla\phi_i\cdot \bm{\phi}_j \,d\bm{x}
    &\approx \sum_{q=1}^{n_b}
    \left( \frac{\partial \widehat{\phi_i}}{\partial \xi}(\bm{\xi}^n_q) J^{-1}_{11}(\bm{x}^n_q)
    + \frac{\partial \widehat{\phi_i}}{\partial \eta}(\bm{\xi}^n_q) J^{-1}_{12}(\bm{x}^n_q) \right)
    \phi_j(\bm{\xi}^n_q) |J|\bigg|_{\bm{x}^n_q} \\
    & \qquad + \sum_{q=1}^{n_b}
    \left( \frac{\partial \widehat{\phi_i}}{\partial \xi}(\bm{\xi}^n_q) J^{-1}_{21}(\bm{x}^n_q)
    + \frac{\partial \widehat{\phi_i}}{\partial \eta}(\bm{\xi}^n_q) J^{-1}_{22}(\bm{x}^n_q) \right)
    \phi_j(\bm{\xi}^n_q) |J|\bigg|_{\bm{x}^n_q}
    \end{split}
    \label{eq:disc_weak_laplacian_qf_mat}
  \end{align}

  Where we emphasize that there are no weights and all quantities of interest are evaluated at nodal
  points $\bm{\xi}^n_q$ or $\bm{x}^n_q$ rather than quadrature points. This works acceptably for
  straight-sided elements, but does it work for curved elements? by analogy, we create the
  quadrature free weak laplacian operator. \\

  Weak laplacian (analogy):
  \begin{align}
    \begin{split}
    &\int_{K}^{} \nabla\phi_j\cdot\nabla\phi_i \,d\bm{x} 
    \approx
    \sum_{q=1}^{n_b}
    \left( \frac{\partial \widehat{\phi_i}}{\partial \xi}(\bm{\xi}^n_q) J^{-1}_{11}(\bm{x}^n_q)
    + \frac{\partial \widehat{\phi_i}}{\partial \eta}(\bm{\xi}^n_q) J^{-1}_{12}(\bm{x}^n_q) \right) 
    \left( \frac{\partial \widehat{\phi_j}}{\partial \xi}(\bm{\xi}^n_q) J^{-1}_{11}(\bm{x}^n_q)
    + \frac{\partial \widehat{\phi_j}}{\partial \eta}(\bm{\xi}^n_q) J^{-1}_{12}(\bm{x}^n_q) \right) 
    |J|\bigg|_{\bm{x}^n_q}\\
    &\qquad+ 
    \sum_{q=1}^{n_b}
    \left( \frac{\partial \widehat{\phi_i}}{\partial \xi}(\bm{\xi}^n_q) J^{-1}_{21}(\bm{x}^n_q)
    + \frac{\partial \widehat{\phi_i}}{\partial \eta}(\bm{\xi}^n_q) J^{-1}_{22}(\bm{x}^n_q) \right) 
    \left( \frac{\partial \widehat{\phi_j}}{\partial \xi}(\bm{\xi}^n_q) J^{-1}_{21}(\bm{x}^n_q)
    + \frac{\partial \widehat{\phi_j}}{\partial \eta}(\bm{\xi}^n_q) J^{-1}_{22}(\bm{x}^n_q) \right) 
    |J|\bigg|_{\bm{x}^n_q}
    \end{split}
    \label{eq:disc_weak_laplacian_qf}
  \end{align}

\subsection{discrete differentiation operators}

  It is often the case that we wish to take derivatives of fields explicitly, in the case of source
  terms and the like. In this case, we can take advantage of the fact that we have computed the
  $\frac{\partial \bm{xi}}{\partial \bm{x}}$ $J^{-1}$ operators, which is discussed in
  \cite{hesthaven2007nodal}, in chapter 6. Since we suppose that the transformation
  $\bm{x}(\bm{\xi})$ is invertible, with inverse transformation $\bm{\xi}(\bm{x})$. Then 
  (\ref{eq:def_inv_der}) can be used to assert

  \begin{align}
    \begin{split}
      &\frac{\partial \phi}{\partial x} = 
        \frac{\partial \phi}{\partial \xi} \frac{\partial \xi}{\partial x}
        + \frac{\partial \phi}{\partial \eta} \frac{\partial \eta}{\partial x}
        \qquad
      \frac{\partial \phi}{\partial y} = 
        \frac{\partial \phi}{\partial \xi} \frac{\partial \xi}{\partial y}
        + \frac{\partial \phi}{\partial \eta} \frac{\partial \eta}{\partial y} \\
      \Rightarrow &\frac{\partial }{\partial x}(\cdot) 
      = \frac{\partial \xi}{\partial x} \mathcal{D}_{\xi} + \frac{\partial \eta}{\partial x}
      \mathcal{D}_{\eta} \qquad
      \frac{\partial }{\partial x}(\cdot) 
      =\frac{\partial \xi}{\partial y} \mathcal{D}_{\xi} + \frac{\partial \eta}{\partial y}
      \mathcal{D}_{\eta}
    \end{split}
  \end{align}

  Where $\mathcal{D}_k$ is the discrete differentiation matrix $MS_k$ in coordinate direction
  $k$ defined in \cite{hesthaven2007nodal}. Now that we have defined these discrete operators, we
  can mix and match them in order to take gradients, divergences, curls etc. of the scalar or vector
  fields in question.

% bibliography
\nocite{*}
\bibliography{isoparametric_mappings.bib}{
  \bibliographystyle{plain}}
\end{document}
